\documentclass[12pt]{article}
\usepackage[french]{babel}
\usepackage[utf8]{inputenc}
\usepackage[T1]{fontenc}
\usepackage{graphicx}
\usepackage{fancyhdr}
\usepackage{amsmath}
\setlength{\headheight}{36.0976pt}
\addtolength{\topmargin}{-21.0976pt}
\usepackage{pifont}
\usepackage{geometry}
\usepackage{listings}

\usepackage{ragged2e}  % Justify text


%Sousligne
\usepackage{ulem}
\usepackage{amsmath}
\usepackage{matlab-prettifier}
\usepackage{mathtools, bm}
\usepackage{amssymb, bm}

% Commandes pour vecteurs et matrices
\newcommand{\mat}[1]{\mathbf{#1}}
\newcommand{\vect}[1]{\boldsymbol{#1}}

\usepackage{xcolor}
\makeatletter
\let\ps@plain\ps@fancy
\makeatother
\def\ind{\textrm{1\kern-0.25emI}}
\definecolor{backcolour}{rgb}{0.95,0.95,0.92}
\definecolor{codegreen}{rgb}{0,0.6,0}
\lstdefinestyle{mystyle}{
    backgroundcolor=\color{backcolour},
    keywordstyle=\color{blue},
    commentstyle=\color{codegreen},
    numberstyle=\tiny\color{black},
    frame=single,
    breakatwhitespace=false,
    breaklines=true,
    captionpos=b,
    keepspaces=true,
    numbers=left,
    numbersep=5pt,
    showspaces=false,
    showstringspaces=false,
    showtabs=false,
    tabsize=2
}
\lstset{style=mystyle}
\usepackage{amsmath}
%Sousligne
\usepackage{ulem}
%Circuit elec
\usepackage[european, straightvoltages]{circuitikz}


\usepackage{tcolorbox}
\newtcolorbox{definitionbox}[1][]{%
  colback=white,
  colframe=subsectioncolor,
  boxrule=0.5pt,
  arc=0pt,
  outer arc=0pt,
  left=2mm,
  right=2mm,
  top=1mm,
  bottom=1mm,
  title=#1
}

\newtcolorbox{theorembox}[1][]{%
  colback=white,
  colframe=sectioncolor,
  boxrule=0.5pt,
  arc=0pt,
  outer arc=0pt,
  left=2mm,
  right=2mm,
  top=1mm,
  bottom=1mm,
  title=#1
}


\usepackage[linesnumbered,ruled,vlined]{algorithm2e}
\SetKwInput{KwIn}{Entrée}
\SetKwInput{KwOut}{Sortie}

%test
%%%%%%%%%%%%%%%%%% TABLE DES MATIERES %%%%%%%
\usepackage{tocloft} % Pour la table des matieres
%Configuration de la table des matières
\renewcommand{\cftsecfont}
{\color{sectioncolor}\bfseries}
\renewcommand{\cftsecpagefont}{\color{sectioncolor}\bfseries}
\setlength{\cftsecnumwidth}{2.5em}
\cftsetindents{section}{0em}{3em}

% Pour les sous-sections
\renewcommand{\cftsubsecfont}{\color{subsectioncolor}\bfseries}
\renewcommand{\cftsubsecpagefont}{\color{subsectioncolor}\bfseries}
\setlength{\cftsubsecnumwidth}{3em}


% Pour les sous-sous-sections (même couleur, sans gras)
\renewcommand{\cftsubsubsecfont}{\color{subsectioncolor}}
\cftsetindents{subsubsection}{4em}{4.5em}
\renewcommand{\cftsubsubsecpagefont}{\color{subsectioncolor}}
\cftsetindents{subsection}{2em}{4em}
%\setlength{\cftsubsubsecnumwidth}{3.5em}


%%%%% PARTIES
\renewcommand{\cftpartfont}{\color{sectioncolor}\bfseries\Large}
\renewcommand{\cftpartpagefont}
{\color{sectioncolor}\bfseries\Large}
\cftsetindents{part}{0em}{4em}

\renewcommand{\cftbeforepartskip}{1em}
\renewcommand{\cftpartfont}{%
  \color{sectioncolor}\bfseries\Large%
  \rule{\linewidth}{0.4pt}\\%
}

\renewcommand{\cftpartpresnum}{}
\setlength{\cftpartnumwidth}{0em}
\renewcommand{\thepart}{}


\usepackage{hyperref}
\hypersetup{
    %colorlinks=true,
    linktoc=all,
    %linkcolor=sectioncolor,
    filecolor=sectioncolor,
    urlcolor=subsectioncolor,
    pdfborder={0 0 0},
}

%Pour titre plus coool
\usepackage{mdframed}
\usepackage[utf8]{inputenc}
\usepackage{booktabs} % Pour de plus jolies tableaux
\usepackage{titlesec}
% Configuration de la géométrie de la page
\geometry{a4paper, top=2cm, bottom=2cm, left=2cm, right=2cm, headheight=15pt, includeheadfoot}
% Configuration des en-têtes et des pieds de page
\pagestyle{fancy}
\fancyhf{}
\fancyhead[R]{\includegraphics[width=2.5cm]{images.png}}

%Configuration des parties
\titleformat{\part}[display]
    {\Huge\bfseries\centering\color{sectioncolor}}
    {}
    {0pt}
    {\rule{\textwidth}{1pt}\\\vspace{0.2em}}
    [\vspace{0.2em}\rule{\textwidth}{1pt}]

\titlespacing*{\part}
    {0pt}  % Espacement gauche
    {-60pt} % Espacement avant (valeur négative pour monter)
    {20pt}  % Espacement après


%%%%%%%%%% NUMERA UE %%%%%%%
\fancyhead[L]{Identification par Moindres Carrés}

%%%%%%%%%%%% Nom matière

\fancyfoot[L]{Gatien Séguy \& Maxime Degraeve}

\fancyfoot[R]{\thepage}
\renewcommand{\headrulewidth}{0.4pt}
\renewcommand{\footrulewidth}{0.4pt}

\usepackage{xcolor}
\definecolor{sectioncolor}{rgb}{0.7, 0.1, 0.1} % Définir la couleur rouge

% Configuration des titres de sections

\titleformat{\section}
{\normalfont\Large\bfseries\color{sectioncolor}}{\thesection}{1em}{}
%{\normalfont\large\bfseries\color{sectioncolor}\MakeUppercase}{\thesection}{1em}{}

\titlespacing*{\section}{0pt}{3.5ex plus 1ex minus .2ex}{2.3ex plus .2ex}

\renewcommand{\thesection}{\Roman{section}}



%Configuration des sous titre
\usepackage{xcolor}
\definecolor{subsectioncolor}{rgb}{0, 0.51, 0.58}
% Configuration des titres des sous-sections
\titleformat{\subsection}
{\normalfont\large\bfseries\color{subsectioncolor}}{\thesubsection}{1em}{}

% Configuration des titres des sous-sous-sections
\titleformat{\subsubsection}
{\normalfont\normalsize\bfseries\color{subsectioncolor}}{\thesubsubsection}{1em}{}

\begin{comment} % BLASON 1
\usepackage{background}
\newcommand{\blasonpage}{%
  \backgroundsetup{
    scale=0.5,
    angle=0,
    opacity=0.05,
    contents={\includegraphics[width=1.2\paperwidth]{CH_ZH_logo_Kanton_court.png}}
  }
  \BgThispage
}
\end{comment}

\begin{document}
\fontsize{13pt}{15pt}\selectfont

\begin{titlepage}
\thispagestyle{empty}
%\blasonpage % BLASON 2
\begin{center}
    % ESPACE HAUT
    %\vspace*{0.8cm}

    % LES 4 LOGOS EN UNE LIGNE
    \begin{minipage}{0.48\textwidth}
        \centering
        \includegraphics[width=\linewidth]{Logo_ENS_leger(1).jpg}
    \end{minipage}
    \hfill
    \begin{minipage}{0.48\textwidth}
        \vspace{-0.4cm}
        \includegraphics[width=\linewidth]{logo_univPS.png}
    \end{minipage}

    \vspace{1.4cm}

    {\Large \textbf{ÉCOLE NORMALE SUPÉRIEURE PARIS-SACLAY}}\\
    {\small (Université Paris-Saclay)}\\[0.2cm]


        \vspace{0.3cm}
    \noindent\makebox[\textwidth]{\color{black}\rule{1\textwidth}{2pt}}\\[-1pt]
    \noindent\makebox[\textwidth]{\color{black}\rule{0.8\textwidth}{1pt}}
    \vspace{0.3cm}


    {\Huge  \color{sectioncolor} \textbf{Compte rendu de TP}}\\[0.2cm]

{\large \textbf{Matière :} \itshape Automatique non-linéaire et Filtrage de Kalman }\\[0.2cm]


    %%%% TRAIT
    \vspace{0.3cm}
    \noindent\makebox[\textwidth]{\color{black}\rule{0.8\textwidth}{1pt}}\\[-1pt]
    \noindent\makebox[\textwidth]{\color{black}\rule{1\textwidth}{2pt}}
    \vspace{0.3cm}

    %Fin Trait
    \vspace{0.1cm}
    {\LARGE \textbf{TP2 - Éstimation de la position et de la vitesse d'un mobile par filtrage de Kalman}}\\[1cm]

    \begin{tabular}{rl}
    \textbf{Nom de l'étudiant :} & Gatien Séguy \& Maxime Degraeve \\
    \textbf{Établissement :} & ENS Paris-Saclay (Département EEA) \\
    \textbf{Encadrante :} & Jean-Pierre Barbot\\
    \textbf{Date :} & \today \\
    \end{tabular}

    \vfill

\end{center}
\end{titlepage}
% \backgroundsetup{contents={}} % BLASON 3
% Sommaire

\newpage
\vspace*{0.5cm}
\tableofcontents
\newpage


% =============================================================================
% INTRODUCTION GÉNÉRALE
% =============================================================================
\section{Introduction}

Dans ce TP, on considère un mobile se déplaçant en mouvement rectiligne quasi-uniforme. L'objectif est d'estimer sa position et sa vitesse (constituant son état) à partir de la seule mesure bruitée de sa position. Le caractère "quasi-uniforme" du mouvement se traduit par une accélération non nulle mais de faible amplitude, modélisée comme un bruit blanc.

\bigskip 

Ce problème est représentatif de nombreuses applications pratiques : suivi de cibles radar, localisation GPS, ou encore navigation de véhicules autonomes.

\bigskip 

Les objectifs de ce travail pratique sont les suivants :
\begin{itemize}
    \item Mettre en équation le système sous forme d'un modèle d'état à temps discret, en identifiant les matrices du système et les statistiques des bruits.
    \item Implémenter le filtre de Kalman et illustrer son fonctionnement sur des données simulées.
    \item Étudier l'influence de différents paramètres sur les performances du filtre : nature de la densité de probabilité des bruits, initialisation du filtre, et erreurs de modélisation sur les matrices de covariance.
\end{itemize}

\bigskip

On adopte les notations suivantes tout au long du CR (je prefère radoter pour pas me perdre dans les notations / indices) :
\begin{itemize}
    \item $t_k = kT$ : instants d'échantillonnage avec $T > 0$ la période d'échantillonnage
    \item $d_k$ : position du mobile à l'instant $t_k$
    \item $v_k$ : vitesse du mobile à l'instant $t_k$
    \item $\gamma_k$ : accélération du mobile sur l'intervalle $[t_k, t_{k+1}]$
    \item $\mat{x}_k = (d_k \; v_k)^T$ : vecteur d'état
    \item $y_k$ : mesure de la position à l'instant $t_k$
    \item $w_k$ : bruit de mesure
    \item $\sigma_\gamma$ : écart-type de l'accélération ($\sigma_\gamma = 0{,}01$ m/s$^2$)
    \item $\sigma_d$ : écart-type du bruit de mesure ($\sigma_d = 0{,}1$ m)
\end{itemize}


\section{Modèle du système}
\subsection{Préparation 1}

\subsubsection{Dynamique continue du système}

On considère un mobile en mouvement rectiligne. Pour $t \geq kT$, les équations de la cinématique donnent :
\[
\begin{cases}
v(t) = v_k + \displaystyle\int_{kT}^{t} \gamma(\tau)\, d\tau \\[2ex]
d(t) = d_k + \displaystyle\int_{kT}^{t} v(\tau)\, d\tau
\end{cases}
\]

\subsubsection{Discrétisation avec accélération constante par morceaux}

Entre deux instants d'échantillonnage consécutifs, l'accélération est supposée constante : pour $t \in [kT, (k+1)T]$, on a $\gamma(t) = \gamma_k$.

En intégrant, on obtient l'évolution de la vitesse :
\[
v(t) = v_k + \gamma_k (t - kT)
\]

Puis, en intégrant la vitesse, on obtient l'évolution de la position :
\[
d(t) = d_k + \int_{kT}^{t} \left[ v_k + \gamma_k (\tau - kT) \right] d\tau = d_k + v_k (t - kT) + \frac{\gamma_k}{2} (t - kT)^2
\]

\subsubsection{Équations aux instants d'échantillonnage}

En évaluant ces expressions à l'instant $t = (k+1)T$, on obtient les équations récurrentes :
\[
\boxed{
\begin{cases}
d_{k+1} = d_k + v_k T + \dfrac{\gamma_k T^2}{2} \\[2ex]
v_{k+1} = v_k + \gamma_k T
\end{cases}
}
\]

\subsubsection{Représentation d'état}

En définissant le vecteur d'état $\mat{x}_k = \begin{pmatrix} d_k \\ v_k \end{pmatrix}$, on peut écrire le système sous forme matricielle :
\[
\mat{x}_{k+1} = \underbrace{\begin{pmatrix} 1 & T \\ 0 & 1 \end{pmatrix}}_{\mat{A}} \mat{x}_k + \underbrace{\begin{pmatrix} \frac{T^2}{2} \\[1ex] T \end{pmatrix}}_{\mat{B}} \gamma_k
\]

Le bruit d'état est donc $\mat{b}_k = \mat{B} \gamma_k$ où $\gamma_k$ est un bruit blanc de variance $\sigma_\gamma^2$.

\subsubsection{Matrice de covariance du bruit d'état}

La matrice de covariance du bruit d'état se calcule comme suit :
\[
\mat{Q} = \mathbb{E}\left[\mat{b}_k \mat{b}_k^T\right] = \mathbb{E}\left[\mat{B} \gamma_k \gamma_k^T \mat{B}^T\right] = \mat{B} \, \mathbb{E}\left[\gamma_k^2\right] \mat{B}^T = \sigma_\gamma^2 \, \mat{B} \mat{B}^T
\]

En développant le produit $\mat{B} \mat{B}^T$ :
\[
\mat{B} \mat{B}^T = \begin{pmatrix} \frac{T^2}{2} \\[1ex] T \end{pmatrix} \begin{pmatrix} \frac{T^2}{2} & T \end{pmatrix} = \begin{pmatrix} \frac{T^4}{4} & \frac{T^3}{2} \\[1ex] \frac{T^3}{2} & T^2 \end{pmatrix}
\]

D'où la matrice de covariance du bruit d'état :
\[
\boxed{
\mat{Q} = \sigma_\gamma^2 \begin{pmatrix} \frac{T^4}{4} & \frac{T^3}{2} \\[1ex] \frac{T^3}{2} & T^2 \end{pmatrix}
}
\]

\subsubsection{Équation d'observation}

La mesure $y_k$ correspond à la position bruitée du mobile :
\[
y_k = d_k + w_k = \underbrace{\begin{pmatrix} 1 & 0 \end{pmatrix}}_{\mat{C}} \mat{x}_k + w_k
\]

où $w_k$ est un bruit blanc de variance $\sigma_d^2$.

La matrice de covariance du bruit d'observation est donc :
\[
\boxed{R = \sigma_d^2}
\]

\subsubsection{Récapitulatif du modèle d'état}

Le système à temps discret s'écrit sous la forme canonique :
\[
\boxed{
\begin{cases}
\mat{x}_{k+1} = \mat{A} \mat{x}_k + \mat{b}_k \\[1ex]
y_k = \mat{C} \mat{x}_k + w_k
\end{cases}
}
\]

avec les matrices et les statistiques des bruits :
\[
\mat{A} = \begin{pmatrix} 1 & T \\ 0 & 1 \end{pmatrix}, \quad
\mat{C} = \begin{pmatrix} 1 & 0 \end{pmatrix}
\]
\[
\mathbb{E}[\mat{b}_k] = \mat{0}, \quad \mathbb{E}[\mat{b}_k \mat{b}_k^T] = \mat{Q}, \quad
\mathbb{E}[w_k] = 0, \quad \mathbb{E}[w_k^2] = R
\]

\section{Éstimation de l'état du système}
\subsection{Préparation 2}



\subsection{Manipulation}

\subsubsection{Structure generale du programme \texttt{exemple\_FK.m}}

Le programme principal \texttt{exemple\_FK.m} est organise sous forme de menu interactif proposant huit options d'etude. Il est accompagne de trois fonctions auxiliaires :
\begin{itemize}
    \item \texttt{nuage\_n.m} : etude statistique avec bruits gaussiens
    \item \texttt{nuage\_u.m} : etude statistique avec bruits uniformes
    \item \texttt{trace\_ellipse\_P.m} : trace d'ellipses de covariance
\end{itemize}


La premiere partie du programme definit les parametres du modele :

\begin{lstlisting}[language=Matlab]
T = 1;                    % Periode d'echantillonnage
v_0 = 1; m_0 = [0;v_0];   % Etat initial moyen
P_0 = 100*I;              % Covariance initiale

A = [1 T; 0 1];           % Matrice d'etat
V = [T^2/2; T];           % Matrice du bruit d'etat
std_g = 0.01;             % Ecart-type acceleration
Q = std_g^2*V*V';         % Covariance bruit d'etat

C = [1 0];                % Matrice d'observation
std_w = 0.1*v_0*T;        % Ecart-type bruit de mesure
R = std_w^2;              % Covariance bruit d'observation
\end{lstlisting}

\textbf{Objectif :} Definir le modele d'etat conformement aux equations de la preparation 1, avec les valeurs numeriques specifiees dans l'enonce ($T = 1$ s, $\sigma_\gamma = 0{,}01$ m/s$^2$, $\sigma_d = 0{,}1$ m).

\bigskip
\hrule
\hrule
\bigskip

\textbf{Option 1 }: Illustration du fonctionnement du filtre de Kalman\\
Cette option realise les etapes suivantes :\\

Simulation de la trajectoire reelle
Generation d'une trajectoire de $k_{\max} = 20$ iterations avec :

\begin{itemize}
    \item Bruits d'etat $\gamma_k$ et d'observation $w_k$ gaussiens
    \item Calcul des observations $y_k = d_k + w_k$
    \item Estimation naive de la vitesse par derivation : $\hat{v}_k = (y_k - y_{k-1})/T$
\end{itemize}

\textit{Implementation du filtre de Kalman}


\begin{lstlisting}[language=Matlab]
for k = 1:k_max-1
    x_pred(:,k+1) = A*x_est(:,k);           % Prediction
    P = A*P*A' + Q;                          % Covariance predite
    K = P*C'*inv(C*P*C' + R);               % Gain de Kalman
    P = (I - K*C)*P;                         % Covariance corrigee
    y_pred(k+1) = C*x_pred(:,k+1);          % Observation predite
    x_est(:,k+1) = x_pred(:,k+1) + K*(y(k+1) - y_pred(k+1)); % Correction
end
\end{lstlisting}

\textbf{Objectif :} Illustrer visuellement le fonctionnement du filtre de Kalman en comparant l'etat reel, l'etat estime par FK et l'estimation naive (derivation). On observe que le FK fournit une estimation bien plus precise, notamment pour la vitesse.

\paragraph{Visualisation des ellipses de confiance}
Le programme trace les ellipses a $3\sigma$ (contenant 99\% de la distribution gaussienne) illustrant :
\begin{itemize}
    \item L'ellipse bleue : precision de l'estimation apres correction
    \item L'ellipse verte pointillee : precision de la prediction sans bruit d'etat
    \item L'ellipse verte : precision de la prediction avec bruit d'etat
    \item La bande jaune : incertitude de l'observation
\end{itemize}

\textbf{Objectif :} Visualiser geometriquement le compromis realise par le filtre entre prediction et observation.

\bigskip
\hrule
\hrule
\bigskip

\textbf{Option 2} : Etude statistique avec bruits gaussiens

Cette option appelle la fonction \texttt{nuage\_n} qui :
\begin{itemize}
    \item Simule $N = 10000$ trajectoires independantes
    \item Calcule l'erreur d'estimation finale $\mat{x}_{k_{\max}} - \hat{\mat{x}}_{k_{\max}}$ pour chaque trajectoire
    \item Trace le nuage de points des erreurs et l'ellipse de covariance calculee par le FK
    \item Compare la covariance empirique avec la covariance theorique
\end{itemize}

\textbf{Objectif :} Valider statistiquement que la matrice de covariance $\mat{P}$ calculee par le filtre de Kalman correspond bien a la dispersion reelle des erreurs d'estimation.

\bigskip
\hrule
\hrule
\bigskip

\textbf{Option 3} : Effet de la densite de probabilite (bruits uniformes)

La fonction \texttt{nuage\_u} realise la meme etude statistique mais avec des bruits uniformes (au lieu de gaussiens) :
\begin{lstlisting}[language=Matlab]
g = std_g*(rand(1,k_max) - 0.5)*2*sqrt(3);  % Bruit uniforme
w = std_w*(rand(1,k_max) - 0.5)*2*sqrt(3);  % meme variance que gaussien
\end{lstlisting}

\textbf{Objectif :} Verifier que le filtre de Kalman reste performant meme lorsque les bruits ne sont pas gaussiens. Le facteur $2\sqrt{3}$ assure que la variance du bruit uniforme est identique a celle du bruit gaussien.

\bigskip
\hrule
\hrule
\bigskip

\textbf{Option 4} : Effet de l'etat initial

Comparaison de deux initialisations differentes :
\begin{itemize}
    \item Initialisation 1 : $\hat{\mat{x}}_0 = (0, 1)^T$ (etat initial reel)
    \item Initialisation 2 : $\hat{\mat{x}}_0 = (0, 0)^T$ (vitesse initiale erronee)
\end{itemize}

\textbf{Objectif :} Montrer que le filtre de Kalman converge vers l'etat reel meme avec une initialisation erronee, grace a l'apport d'information des observations successives.

\bigskip
\hrule
\hrule
\bigskip

\textbf{Option 5} : Effet de la covariance initiale

Comparaison de trois valeurs de covariance initiale :
\begin{itemize}
    \item $\mat{P}_0$ : valeur nominale
    \item $100 \times \mat{P}_0$ : grande incertitude initiale (surestimation)
    \item $0{,}01 \times \mat{P}_0$ : faible incertitude initiale (sous-estimation)
\end{itemize}

\textbf{Objectif :} Etudier l'influence de $\mat{P}_0$ sur la convergence du filtre. Une surestimation de $\mat{P}_0$ donne plus de poids aux observations et accelere la convergence. Une sous-estimation peut ralentir la convergence car le filtre "fait trop confiance" a sa prediction.

\bigskip
\hrule
\hrule
\bigskip

\textbf{Option 6} : Effet de l'erreur sur la covariance du bruit d'etat $\mat{Q}$

Comparaison de trois valeurs de $\mat{Q}$ utilisees dans le filtre :
\begin{itemize}
    \item $\mat{Q}$ : valeur correcte
    \item $10 \times \mat{Q}$ : surestimation du bruit d'etat
    \item $0{,}1 \times \mat{Q}$ : sous-estimation du bruit d'etat
\end{itemize}

\textbf{Objectif :} Analyser les consequences d'une erreur de modelisation sur $\mat{Q}$ :
\begin{itemize}
    \item \textbf{Surestimation} ($10\mat{Q}$) : le filtre accorde plus de poids aux observations, l'estimation est plus bruitee mais suit mieux les variations.
    \item \textbf{Sous-estimation} ($0{,}1\mat{Q}$) : le filtre fait trop confiance au modele, l'estimation est plus lisse mais peut presenter un biais.
\end{itemize}

\bigskip
\hrule
\hrule
\bigskip

\textbf{Option 7} : Effet de l'erreur sur la covariance du bruit d'observation $R$

Comparaison de trois valeurs de $R$ utilisees dans le filtre :
\begin{itemize}
    \item $R$ : valeur correcte
    \item $10 \times R$ : surestimation du bruit d'observation
    \item $0{,}1 \times R$ : sous-estimation du bruit d'observation
\end{itemize}

\textbf{Objectif :} Analyser les consequences d'une erreur de modelisation sur $R$ :
\begin{itemize}
    \item \textbf{Surestimation} ($10R$) : le filtre accorde moins de poids aux observations, l'estimation est plus lisse mais peut diverger.
    \item \textbf{Sous-estimation} ($0{,}1R$) : le filtre fait trop confiance aux mesures, l'estimation suit le bruit d'observation.
\end{itemize}

\bigskip
\hrule
\hrule
\bigskip

\textbf{Option 8} : Test de coherence (distance de Mahalanobis)

Cette option calcule la distance de Mahalanobis de l'innovation :
\[
\delta_k = \frac{|y_k - \hat{y}_{k|k-1}|}{\sigma_k}
\]

ou $\sigma_k = \sqrt{\mat{C}\mat{P}_{k|k-1}\mat{C}^T + R}$ est l'ecart-type de l'innovation.

\textbf{Objectif :} Fournir un indicateur de coherence du filtre. Si le modele est correct et les bruits gaussiens, on doit avoir $\Pr[|\delta_k| < 3] \approx 0{,}99$. Une distance regulierement superieure a 3 indique une incoherence entre le modele utilise par le filtre et le systeme reel (erreur sur $\mat{Q}$ ou $R$).

\subsubsection{Fonctions auxiliaires}

\paragraph{Fonction \texttt{trace\_ellipse\_P}}
Trace une ellipse de covariance definie par son centre et sa matrice $\mat{P}$ :
\begin{itemize}
    \item Diagonalisation de $\mat{P}$ pour obtenir les axes principaux
    \item Parametrage de l'ellipse : $\mat{x} = \text{Centre} + \mat{R}\sqrt{\mat{D}} \begin{pmatrix} \cos\theta \\ \sin\theta \end{pmatrix}$
\end{itemize}

\paragraph{Fonctions \texttt{nuage\_n} et \texttt{nuage\_u}}
Realisent une etude Monte-Carlo avec $N$ trajectoires pour valider statistiquement les performances du filtre. La difference reside dans la generation des bruits : gaussiens (\texttt{randn}) ou uniformes (\texttt{rand}).

\end{document}
