\documentclass[12pt]{article}
\usepackage[french]{babel}
\usepackage[utf8]{inputenc}
\usepackage[T1]{fontenc}
\usepackage{graphicx}
\usepackage{fancyhdr}
\usepackage{amsmath}
\setlength{\headheight}{36.0976pt}
\addtolength{\topmargin}{-21.0976pt}
\usepackage{pifont}
\usepackage{geometry}
\usepackage{listings}

\usepackage{ragged2e}  % Justify text


%Sousligne
\usepackage{ulem}
\usepackage{amsmath}
\usepackage{matlab-prettifier}
\usepackage{mathtools, bm}
\usepackage{amssymb, bm}

% Commandes pour vecteurs et matrices
\newcommand{\mat}[1]{\mathbf{#1}}
\newcommand{\vect}[1]{\boldsymbol{#1}}

\usepackage{xcolor}
\makeatletter
\let\ps@plain\ps@fancy
\makeatother
\def\ind{\textrm{1\kern-0.25emI}}
\definecolor{backcolour}{rgb}{0.95,0.95,0.92}
\definecolor{codegreen}{rgb}{0,0.6,0}
\lstdefinestyle{mystyle}{
    backgroundcolor=\color{backcolour},
    keywordstyle=\color{blue},
    commentstyle=\color{codegreen},
    numberstyle=\tiny\color{black},
    frame=single,
    breakatwhitespace=false,
    breaklines=true,
    captionpos=b,
    keepspaces=true,
    numbers=left,
    numbersep=5pt,
    showspaces=false,
    showstringspaces=false,
    showtabs=false,
    tabsize=2
}
\lstset{style=mystyle}
\usepackage{amsmath}
%Sousligne
\usepackage{ulem}
%Circuit elec
\usepackage[european, straightvoltages]{circuitikz}


\usepackage{tcolorbox}
\newtcolorbox{definitionbox}[1][]{%
  colback=white,
  colframe=subsectioncolor,
  boxrule=0.5pt,
  arc=0pt,
  outer arc=0pt,
  left=2mm,
  right=2mm,
  top=1mm,
  bottom=1mm,
  title=#1
}

\newtcolorbox{theorembox}[1][]{%
  colback=white,
  colframe=sectioncolor,
  boxrule=0.5pt,
  arc=0pt,
  outer arc=0pt,
  left=2mm,
  right=2mm,
  top=1mm,
  bottom=1mm,
  title=#1
}


\usepackage[linesnumbered,ruled,vlined]{algorithm2e}
\SetKwInput{KwIn}{Entrée}
\SetKwInput{KwOut}{Sortie}

%test
%%%%%%%%%%%%%%%%%% TABLE DES MATIERES %%%%%%%
\usepackage{tocloft} % Pour la table des matieres
%Configuration de la table des matières
\renewcommand{\cftsecfont}
{\color{sectioncolor}\bfseries}
\renewcommand{\cftsecpagefont}{\color{sectioncolor}\bfseries}
\setlength{\cftsecnumwidth}{2.5em}
\cftsetindents{section}{0em}{3em}

% Pour les sous-sections
\renewcommand{\cftsubsecfont}{\color{subsectioncolor}\bfseries}
\renewcommand{\cftsubsecpagefont}{\color{subsectioncolor}\bfseries}
\setlength{\cftsubsecnumwidth}{3em}


% Pour les sous-sous-sections (même couleur, sans gras)
\renewcommand{\cftsubsubsecfont}{\color{subsectioncolor}}
\cftsetindents{subsubsection}{4em}{4.5em}
\renewcommand{\cftsubsubsecpagefont}{\color{subsectioncolor}}
\cftsetindents{subsection}{2em}{4em}
%\setlength{\cftsubsubsecnumwidth}{3.5em}


%%%%% PARTIES
\renewcommand{\cftpartfont}{\color{sectioncolor}\bfseries\Large}
\renewcommand{\cftpartpagefont}
{\color{sectioncolor}\bfseries\Large}
\cftsetindents{part}{0em}{4em}

\renewcommand{\cftbeforepartskip}{1em}
\renewcommand{\cftpartfont}{%
  \color{sectioncolor}\bfseries\Large%
  \rule{\linewidth}{0.4pt}\\%
}

\renewcommand{\cftpartpresnum}{}
\setlength{\cftpartnumwidth}{0em}
\renewcommand{\thepart}{}


\usepackage{hyperref}
\hypersetup{
    %colorlinks=true,
    linktoc=all,
    %linkcolor=sectioncolor,
    filecolor=sectioncolor,
    urlcolor=subsectioncolor,
    pdfborder={0 0 0},
}

%Pour titre plus coool
\usepackage{mdframed}
\usepackage[utf8]{inputenc}
\usepackage{booktabs} % Pour de plus jolies tableaux
\usepackage{titlesec}
% Configuration de la géométrie de la page
\geometry{a4paper, top=2cm, bottom=2cm, left=2cm, right=2cm, headheight=15pt, includeheadfoot}
% Configuration des en-têtes et des pieds de page
\pagestyle{fancy}
\fancyhf{}
\fancyhead[R]{\includegraphics[width=2.5cm]{images.png}}

%Configuration des parties
\titleformat{\part}[display]
    {\Huge\bfseries\centering\color{sectioncolor}}
    {}
    {0pt}
    {\rule{\textwidth}{1pt}\\\vspace{0.2em}}
    [\vspace{0.2em}\rule{\textwidth}{1pt}]

\titlespacing*{\part}
    {0pt}  % Espacement gauche
    {-60pt} % Espacement avant (valeur négative pour monter)
    {20pt}  % Espacement après


%%%%%%%%%% NUMERA UE %%%%%%%
\fancyhead[L]{Identification par Moindres Carrés}

%%%%%%%%%%%% Nom matière

\fancyfoot[L]{Gatien Séguy \& }

\fancyfoot[R]{\thepage}
\renewcommand{\headrulewidth}{0.4pt}
\renewcommand{\footrulewidth}{0.4pt}

\usepackage{xcolor}
\definecolor{sectioncolor}{rgb}{0.7, 0.1, 0.1} % Définir la couleur rouge

% Configuration des titres de sections

\titleformat{\section}
{\normalfont\Large\bfseries\color{sectioncolor}}{\thesection}{1em}{}
%{\normalfont\large\bfseries\color{sectioncolor}\MakeUppercase}{\thesection}{1em}{}

\titlespacing*{\section}{0pt}{3.5ex plus 1ex minus .2ex}{2.3ex plus .2ex}

\renewcommand{\thesection}{\Roman{section}}



%Configuration des sous titre
\usepackage{xcolor}
\definecolor{subsectioncolor}{rgb}{0, 0.51, 0.58}
% Configuration des titres des sous-sections
\titleformat{\subsection}
{\normalfont\large\bfseries\color{subsectioncolor}}{\thesubsection}{1em}{}

% Configuration des titres des sous-sous-sections
\titleformat{\subsubsection}
{\normalfont\normalsize\bfseries\color{subsectioncolor}}{\thesubsubsection}{1em}{}

\begin{comment} % BLASON 1
\usepackage{background}
\newcommand{\blasonpage}{%
  \backgroundsetup{
    scale=0.5,
    angle=0,
    opacity=0.05,
    contents={\includegraphics[width=1.2\paperwidth]{CH_ZH_logo_Kanton_court.png}}
  }
  \BgThispage
}
\end{comment}

\begin{document}
\fontsize{13pt}{15pt}\selectfont

\begin{titlepage}
\thispagestyle{empty}
%\blasonpage % BLASON 2
\begin{center}
    % ESPACE HAUT
    %\vspace*{0.8cm}

    % LES 4 LOGOS EN UNE LIGNE
    \begin{minipage}{0.48\textwidth}
        \centering
        \includegraphics[width=\linewidth]{Logo_ENS_leger(1).jpg}
    \end{minipage}
    \hfill
    \begin{minipage}{0.48\textwidth}
        \vspace{-0.4cm}
        \includegraphics[width=\linewidth]{logo_univPS.png}
    \end{minipage}

    \vspace{1.4cm}

    {\Large \textbf{ÉCOLE NORMALE SUPÉRIEURE PARIS-SACLAY}}\\
    {\small (Université Paris-Saclay)}\\[0.2cm]


        \vspace{0.3cm}
    \noindent\makebox[\textwidth]{\color{black}\rule{1\textwidth}{2pt}}\\[-1pt]
    \noindent\makebox[\textwidth]{\color{black}\rule{0.8\textwidth}{1pt}}
    \vspace{0.3cm}


    {\Huge  \color{sectioncolor} \textbf{Compte rendu de TP}}\\[0.2cm]

{\large \textbf{Matière :} \itshape Automatique non-linéaire et Filtrage de Kalman }\\[0.2cm]


    %%%% TRAIT
    \vspace{0.3cm}
    \noindent\makebox[\textwidth]{\color{black}\rule{0.8\textwidth}{1pt}}\\[-1pt]
    \noindent\makebox[\textwidth]{\color{black}\rule{1\textwidth}{2pt}}
    \vspace{0.3cm}

    %Fin Trait
    \vspace{0.1cm}
    {\LARGE \textbf{TP2 - Éstimation de la position et de la vitesse d'un mobile par filtrage de Kalman}}\\[1cm]

    \begin{tabular}{rl}
    \textbf{Nom de l'étudiant :} & Gatien Séguy \& Maxime Degraeve \\
    \textbf{Établissement :} & ENS Paris-Saclay (Département EEA) \\
    \textbf{Encadrante :} & Jean-Pierre Barbot\\
    \textbf{Date :} & \today \\
    \end{tabular}

    \vfill

\end{center}
\end{titlepage}
% \backgroundsetup{contents={}} % BLASON 3
% Sommaire

\newpage
\vspace*{0.5cm}
\tableofcontents
\newpage


% =============================================================================
% INTRODUCTION GÉNÉRALE
% =============================================================================
\section{Introduction}


\subsection{Préparation 1}
Pour $t\geq kT$, on a : 

\[
\begin{cases}
\sigma(t) = v_k + \int_{kT}^{t} \gamma(\tau)\, d\tau\\
d(t) = d_k + \int_{kT}^{t} v(\tau)\, d\tau
\end{cases}
\]

Si $kT \geq t \geq (k+A)T$ : 
\(
\gamma (\tau) = \gamma_k 
\)
D'où : \(v(t) = v_k + \gamma_k(t-kT) \)
et aux instant d'échantillonnage :
\[
\begin{cases}
v_{k+1} = v_k + \gamma_kT \, | \, \text{ pour } t=(k+1)T \\
d_{k+1} = d_k + \frac{\gamma_k}{2}T^2
\end{cases}
\]

Le vecteur d'état : \\
\[
x_{k+1} = \begin{pmatrix}
d_{k+1}\\
v_{k+1}
\end{pmatrix}
= \begin{pmatrix}
1 & T \\
0 & 1     
\end{pmatrix}
\begin{pmatrix}
d_k \\
v_k
\end{pmatrix}
+ 
0 u_k + b_k \; \; | \; \; \text{ avec } b_k = \begin{pmatrix}
\frac{T^2}{2} \\
T
\end{pmatrix}
 \gamma_k = B \gamma_k
\]

\[
\mathbb Q = \mathbb E [b_k \cdot b_k^T] = \begin{pmatrix}
    T^2 /2 \\ 
    T
\end{pmatrix}
\begin{pmatrix}
    T^2/2 & T
\end{pmatrix}
\sigma_{\gamma}
\]

D'où :\\
\[
\boxed{
    \mathbb Q = \begin{pmatrix}
        \frac{T^2}{2} & \frac{T^3}{2} \\
        \frac{T^3}{2} & \frac{T^2}{2}
    \end{pmatrix}
}
\]
On a également
\[
\begin{cases}
x_{k+1} = A x_k + b_k \\
y_{k+1} = d_k + w_k = C_k x_k + w_k = \begin{pmatrix}
    1 & 0
\end{pmatrix}
x_k + w_k
\end{cases}
\]
\[
x_k = \begin{pmatrix}
    d_k \\
    v_k
\end{pmatrix}
\]
\end{document}
